\documentclass[c, 11pt, handout, xcolor=table]{beamer}
\setbeamertemplate{caption}{\raggedright\insertcaption\par}

\usepackage{default}

\usepackage[utf8]{inputenc}
\usepackage[T1]{fontenc}
\usepackage[frenchb]{babel}
\usepackage{lmodern}
\usepackage[autolanguage]{numprint}
% \usepackage{enumitem}
\usepackage{array}
\usepackage{tabularx} \newcolumntype{C}{>{\centering}X}
\usepackage{multirow}
\usepackage{hhline}
\usepackage{collcell}
\usepackage{subcaption}



\usepackage{amsthm}
\usepackage{amsmath}
\usepackage{amssymb}
\usepackage{mathrsfs}
\usepackage{amsopn}
\usepackage{stmaryrd}


\usepackage{tikz}
\usetikzlibrary{babel,shapes,patterns}


\DeclareCaptionLabelFormat{listing}{Listing #2}
\DeclareCaptionSubType*[arabic]{table}

\usepackage[langlinenos=true,newfloat=true]{minted}
\newcommand{\source}[5]{
  \begin{table}[H]
    \centering
    \inputminted[frame=lines,linenos,style=colorful,fontfamily=tt,breaklines,autogobble,firstline=#3,firstnumber=#3,lastline=#4,label={#2[#3--#4]}]{#1}{../interim/sledge/#2}
    \captionsetup{name=Listing,labelformat=listing,labelsep=endash}
    \caption{#5}
  \end{table}
  }
\newcommand{\sourceC}[4]{\source{C}{#1}{#2}{#3}{#4}}
\newcommand{\sourceLisp}[4]{\source{Lisp}{os/#1}{#2}{#3}{#4}}
\newcommand{\codeC}[1]{\mintinline[style=colorful,fontfamily=tt]{C}{#1}}
\newcommand{\codeLisp}[1]{\mintinline[style=colorful,fontfamily=tt]{Lisp}{#1}}
\newcommand{\codeASM}[1]{\mintinline[style=colorful,fontfamily=tt]{nasm}{#1}}
\newcommand{\code}[1]{\texttt{#1}}
\newcommand{\foreign}[1]{\emph{#1}}


\usetheme{Warsaw}


\title{Projet avancé : étude d'\foreign{Interim OS}}
\author{Marc Ducret \and Florentin Guth}
\date{9 janvier 2017}

% \theoremstyle{definition}
% \newtheorem{definition}{Définition}[section]




\makeatletter
    \newenvironment{withoutheadline}{
        \setbeamertemplate{headline}[default]
        \def\beamer@entrycode{\vspace*{-\headheight}}
    }{}
\makeatother

\setcounter{tocdepth}{1} % Enlève les subsections de la table des matières

\begin{document}

\begin{withoutheadline}
 \begin{frame}
  \titlepage
  \tableofcontents
 \end{frame}
\end{withoutheadline}

\section{Structure du système d'exploitation}

\subsection{Architecture générale}

\begin{frame}
 \frametitle{Différents niveaux}
 
 \begin{block}{2 couches}
  \foreign{Interim OS} est scindé en deux parties : un partie permettant de gérer précisément la mémoire, écrite en \foreign{C}, et une partie contenant la gestion des différents programmes (comme un éditeur, une console, \ldots).
 \end{block}
\end{frame}


\subsection{Section écrite en C}

\begin{frame}
 \frametitle{Fonctionnement général}
 
 La partie gérant la mémoire fonctionne de la manière suivante :
 \begin{itemize}
  \item On lit l'expression \foreign{Minilisp} donnée,
  \item On formate cette expression en une représentation plus structurée pour faciliter la compilation,
  \item On produit du code assembleur correspondant à l'exécution de l'expression donnée, que l'on stocke dans un fichier temporaire,
  \item Le code produit utilise des fonctions spéciales d'allocations qui permettent de faire fonctionner le ramasse-miettes pour libérer la mémoire lorsque c'est nécessaire,
  \item On exécute ce code assembleur,
  \item On affiche le résultat (qui est une valeur \foreign{Minilisp}).
 \end{itemize}
\end{frame}

\subsection{Section écrite en Minilisp}

\begin{frame}
 \frametitle{TODO}
 Si tu pouvais me rédiger (au moins un brouillon) cette partie, ce serait pas mal parce que je pense que tu connais mieux que moi.
\end{frame}


\section{Gestion de la mémoire}

\subsection{Représentation des valeurs Minilisp}

\begin{frame}
 \frametitle{Structure de données}
 \sourceC{minilisp.h}{81}{91}{La \codeC{struct Cell}}
\end{frame}

\begin{frame}
 \frametitle{Avantages}
 Cette représentation a plusieurs avantages :
 \begin{itemize}
  \item Elle permet d'effectuer facilement (en théorie \ldots) la \foreign{garbage collection}, car en fonction du \codeC{tag} on sait si on a affaire à des entiers ou des pointeurs,
  \item Toute les valeurs ont la même taille (sans compter les cellules pointées par l'un des membres), et il n'y a pas de tableau, ce qui simplifie la gestion des blocs en mémoire.
 \end{itemize}
\end{frame}


\subsection{Allocation de la mémoire}

\begin{frame}[fragile]
 \frametitle{Schéma de la mémoire}
 
 \begin{tikzpicture}[arrow/.style={->, ultra thick, rounded corners, line cap=round},
                     pointer/.style={circle, draw, fill = black, inner sep = 0.5mm},
                     scale=0.5]
  \foreach \z in {0, 4} {
   \foreach \y in {0, ..., 8} \draw (\z,     \y) -- +(3, 0);
   \foreach \x in {0, 3}      \draw (\z + \x, 0) -- +(0, 8);
  };
  
  \draw [arrow] (-1.5, 5.5) node[left]  {\codeC{cells_used}}         -- (0, 5.5);
  \draw [arrow] (8.5,  0.5) node[right] {\codeC{free_list_consumed}} -- (7, 0.5);
  \draw [arrow] (8.5,  2.5) node[right] {\codeC{free_list_avail}}    -- (7, 2.5);
  
  \draw [arrow] (5.5, 0.5) node[pointer] {} -- (3, 0.5);
  \draw [arrow] (5.5, 1.5) node[pointer] {} -- (3.5, 1.5) -- (3.5, 2.5) -- (3, 2.5);
  
  \foreach \y in {1, 3, 4} {
   \fill [pattern = north east lines] (0, \y) rectangle (3, \y + 1);
  };
 \end{tikzpicture}
\end{frame}



\subsection{Le Garbage Collector}

\begin{frame}
 \frametitle{GC}
 
 \foreign{Interim OS} utilise un \foreign{GC mark-and-sweep} classique.
 
 \begin{exampleblock}{\foreign{GC mark-and-sweep}}
  Un \foreign{GC mark-and-sweep} est un \foreign{GC} qui parcourt en profondeur le graphe des \codeC{Cells} en marquant les nœuds rencontrés. Dans une deuxième phase, on libère (\foreign {sweep}) les nœuds non visités en parcourant l'intégralité du tas.
 \end{exampleblock}
 
 Les racines sont la pile, les fichiers ouverts et l'environnement global.

\end{frame}



\section{Compilation à la volée}

\begin{frame}
 \frametitle{TODO}
\end{frame}

\section{Améliorations apportées}

\begin{frame}
 \frametitle{TODO}
\end{frame}






\appendix

\section{Annexe}
\subsection{Remplissage}
\begin{frame}
 \frametitle{Truc}
\end{frame}
 
\end{document}

\documentclass[c, 11pt, handout, xcolor=table]{beamer}
\setbeamertemplate{caption}{\raggedright\insertcaption\par}

\usepackage{default}

\usepackage[utf8]{inputenc}
\usepackage[T1]{fontenc}
\usepackage[frenchb]{babel}
\usepackage{lmodern}
\usepackage[autolanguage]{numprint}
% \usepackage{enumitem}  % Si rajouté, fais bugger les puces des listes
\usepackage{array}
\usepackage{tabularx} \newcolumntype{C}{>{\centering}X}
\usepackage{multirow}
\usepackage{hhline}
\usepackage{collcell}
\usepackage{subcaption}



\usepackage{amsthm}
\usepackage{amsmath}
\usepackage{amssymb}
\usepackage{mathrsfs}
\usepackage{amsopn}
\usepackage{stmaryrd}


\usepackage{tikz}
\usetikzlibrary{babel,shapes,patterns}


\DeclareCaptionLabelFormat{listing}{Listing #2}
\DeclareCaptionSubType*[arabic]{table}

\usepackage[langlinenos=true,newfloat=true]{minted}
\newcommand{\source}[5]{
  \begin{table}[H]
    \centering
    \inputminted[frame=lines,linenos,style=colorful,fontfamily=tt,breaklines,autogobble,firstline=#3,firstnumber=#3,lastline=#4,label={#2[#3--#4]}]{#1}{../interim/sledge/#2}
    \captionsetup{name=Listing,labelformat=listing,labelsep=endash}
    \caption{#5}
  \end{table}
  }
\newcommand{\sourceC}[4]{\source{C}{#1}{#2}{#3}{#4}}
\newcommand{\sourceLisp}[4]{\source{Lisp}{os/#1}{#2}{#3}{#4}}
\newcommand{\codeC}[1]{\mintinline[style=colorful,fontfamily=tt]{C}{#1}}
\newcommand{\codeLisp}[1]{\mintinline[style=colorful,fontfamily=tt]{Lisp}{#1}}
\newcommand{\codeASM}[1]{\mintinline[style=colorful,fontfamily=tt]{nasm}{#1}}
\newcommand{\code}[1]{\texttt{#1}}
\newcommand{\foreign}[1]{\emph{#1}}


\usetheme{Warsaw}


\title{Projet avancé : étude d'\foreign{Interim OS}}
\author{Marc Ducret \and Florentin Guth}
\date{9 janvier 2017}

% \theoremstyle{definition}
% \newtheorem{definition}{Définition}[section]




\makeatletter
    \newenvironment{withoutheadline}{
        \setbeamertemplate{headline}[default]
        \def\beamer@entrycode{\vspace*{-\headheight}}
    }{}
\makeatother

\setcounter{tocdepth}{1} % Enlève les subsections de la table des matières

\begin{document}

\begin{withoutheadline}
 \begin{frame}
  \titlepage
  \tableofcontents
 \end{frame}
\end{withoutheadline}

\section{Structure du système d'exploitation}

\subsection{Architecture générale}

\begin{frame}
 \frametitle{Différents niveaux}
 
 \begin{block}{2 couches}
  \foreign{Interim OS} est scindé en deux parties : un partie permettant de gérer précisément la mémoire, écrite en \foreign{C}, et une partie contenant la gestion des différents programmes (comme un éditeur, une console, \ldots).
 \end{block}
\end{frame}


\subsection{Section écrite en C}

\begin{frame}
 \frametitle{Fonctionnement général}
 
 La partie gérant la mémoire fonctionne de la manière suivante :
 \begin{itemize}
  \item On lit l'expression \foreign{Minilisp} donnée,
  \item On formate cette expression en une représentation plus structurée pour faciliter la compilation,
  \item On produit du code assembleur correspondant à l'exécution de l'expression donnée, que l'on stocke dans un fichier temporaire,
  \item Le code produit utilise des fonctions spéciales d'allocations qui permettent de faire fonctionner le ramasse-miettes pour libérer la mémoire lorsque c'est nécessaire,
  \item On exécute ce code assembleur,
  \item On affiche le résultat (qui est une valeur \foreign{Minilisp}).
 \end{itemize}
\end{frame}

\subsection{Section écrite en Minilisp}

\begin{frame}
 \frametitle{Fonctions présentes}
 
 On a accès à toutes les primitives \foreign{C} qui permettent par exemple de changer la couleur d'un pixel de l'écran, de lancer le \foreign{garbage collector}\ldots
 
 Les entrées/sorties du point de vue de l'OS sont toutes représentées par des \codeC{Filesystems} (souris, clavier, réseau, disque dur, écran). Ceci permet une certaine abstraction pour la gestion de ceux-ci (par exemple : \codeLisp{(load "/framebuffer/width")}).
\end{frame}

\begin{frame}
 \frametitle{Gestion des fenêtres}
 
 Le fichier \code{shell.l} se charge de gérer les fenêtres : il maintient une liste de tâches, chacune représentant une fenêtre.
 
 Il y a une tentative de ne pas tout dessiner à chaque cycle pour ne recalculer uniquement les pixels qui sont modifiés. Le résultat est cependant décevant : souvent certains pixels ne sont pas mis à jour mais devraient l'être. Ainsi, lorsque l'on déplace une fenêtre, l'\foreign{OS} ne recalcule que la fenêtre déplacée alors qu'il faudrait également le faire pour les fenêtres recouvertes par celle-ci.
\end{frame}



\section{Gestion de la mémoire}

\subsection{Représentation des valeurs Minilisp}

\begin{frame}
 \frametitle{Structure de données}
 \sourceC{minilisp.h}{81}{91}{La \codeC{struct Cell}}
\end{frame}

\begin{frame}
 \frametitle{Avantages}
 Cette représentation a plusieurs avantages :
 \begin{itemize}
  \item Elle permet d'effectuer facilement (en théorie\ldots) la \foreign{garbage collection}, car en fonction du \codeC{tag} on sait si on a affaire à des entiers ou des pointeurs,
  \item Toute les valeurs ont la même taille (sans compter les cellules pointées par l'un des membres), et il n'y a pas de tableau, ce qui simplifie la gestion des blocs en mémoire.
 \end{itemize}
\end{frame}


\subsection{Allocation de la mémoire}

\begin{frame}[fragile]
 \frametitle{Schéma de la mémoire}
 
 \begin{tikzpicture}[arrow/.style   = {->, ultra thick, rounded corners, line cap = round},
                     pointer/.style = {circle, draw, fill = black, inner sep = 0.5mm},
                     scale=0.5]
  \foreach \z in {0, 4} {
   \foreach \y in {0, ..., 8} \draw (\z,     \y) -- +(3, 0);
   \foreach \x in {0, 3}      \draw (\z + \x, 0) -- +(0, 8);
  };
  
  \draw [arrow] (-1.5, 5.5) node[left]  {\codeC{cells_used}}         -- (0, 5.5);
  \draw [arrow] (8.5,  0.5) node[right] {\codeC{free_list_consumed}} -- (7, 0.5);
  \draw [arrow] (8.5,  2.5) node[right] {\codeC{free_list_avail}}    -- (7, 2.5);
  
  \draw [arrow] (5.5, 0.5) node[pointer] {} -- (3, 0.5);
  \draw [arrow] (5.5, 1.5) node[pointer] {} -- (3.5, 1.5) -- (3.5, 2.5) -- (3, 2.5);
  
  \foreach \y in {1, 3, 4} {
   \fill [pattern = north east lines] (0, \y) rectangle (3, \y + 1);
  };
 \end{tikzpicture}
\end{frame}



\subsection{Le Garbage Collector}

\begin{frame}
 \frametitle{Fonctionnement du GC}
 
 \foreign{Interim OS} utilise un \foreign{GC mark-and-sweep} classique.
 
 \begin{exampleblock}{\foreign{GC mark-and-sweep}}
  Un \foreign{GC mark-and-sweep} est un \foreign{GC} qui parcourt en profondeur le graphe des \codeC{Cells} en marquant les nœuds rencontrés. Dans une deuxième phase, on libère (\foreign {sweep}) les nœuds non visités en parcourant l'intégralité du tas.
 \end{exampleblock}
 
 Les racines sont composées de :
 \begin{itemize}
  \item la pile,
  \item les fichiers ouverts,
  \item l'environnement global.
 \end{itemize}

\end{frame}

\begin{frame}
 \frametitle{Complexité}
 
 L'allocation d'une cellule s'effectue en $O(1)$.
 
 Le marquage se fait en temps $O(|\mathrm{tas}| + |\mathrm{pile}| + |\mathrm{env}| + |\mathrm{fichiers}|)$ et en espace $\Omega($profondeur du tas$)$. Il est possible de l'effectuer en espace constant à l'aide de l'algorithme \foreign{pointer reversal} qui permet d'encoder la pile de récursion dans le tas lui-même.
 
 Le balayage se fait en temps $O(|\mathrm{tas}|)$, et en espace en plus des deux tableaux précédents $O(1)$.
 
\end{frame}



\section{Compilation à la volée}

\subsection{Primitives C}

\begin{frame}
 \frametitle{Mélange C-assembleur}
 
 Le code s'appuie beaucoup sur le fait que les fonctions écrites en \foreign{C} sont accessibles depuis du code assembleur écrit pendant l'exécution du programme, ce qui permet d'utiliser notamment les fonctions d'allocations directement afin de pouvoir faire appel au \foreign{garbage collector} non seulement pour le code statique mais aussi pour du code écrit dans le \foreign{REPL} par exemple.
\end{frame}

\subsection{La fonction \code{compile\_expr}}

\begin{frame}
 \frametitle{Remarques notables}
 
 On effectue de petites vérifications : définitions des symboles, typage statique et typage dynamique (lors de l'utilisation d'une fonction comme \codeLisp{get8}).
 
 On adopte une stratégie d'\foreign{eager evaluation} pour les appels de fonction (passage par valeur) : tous les arguments sont d'abord évalués, passés par registres (maximum 4) puis sur la pile et enfin la fonction est appelée.
 
 Lorsqu'on compile des définitions de fonctions, on précalcule l'espace de pile nécessaire à chaque fonction pour l'allouer en une seule fois lors de l'appel.
\end{frame}



\section{Améliorations apportées}

\subsection{Optimisation de la garbage collection}

\begin{frame}
 \frametitle{Appels que lorsque nécessaire}
 
 Nous avons remplacé les appels au \foreign{GC} par une nouvelle \foreign{builtin} \codeLisp{gc-if-needed} qui appelle le \foreign{GC} seulement si la mémoire est trop remplie.

 On a obtenu les résultats suivants :
 \begin{description}
  \item [\foreign{GC} systématique :] 60 images en 90 unités de temps
  \item [\foreign{GC} si nécessaire :] 60 images en 60 unités de temps
 \end{description}
\end{frame}

\begin{frame}
 \frametitle{Implémentation}
 
 \tiny
 \sourceC{alloc.c}{186}{191}{Appel du \foreign{GC} si nécessaire}
 \begin{table}[H]
  \centering
  \inputminted[frame=lines,linenos,style=colorful,fontfamily=tt,breaklines,autogobble,firstline=1638,firstnumber=1638,lastline=1646,label={compiler\_new[1638--1646]}]{C}{../interim/sledge/compiler_new.c}
  \captionsetup{name=Listing,labelformat=listing,labelsep=endash}
  \caption{Ajout de la primitive}
 \end{table}
%  \sourceLisp{shell.l}{264}{264}{Appel de la fonction}
\end{frame}

 
\subsection{Système de tâches}

\begin{frame}
 \frametitle{Tâche du bureau}
 
  Pour pallier aux problèmes précédemment cités, on a effectué les ajouts suivants :
 \begin{itemize}
  \item \codeLisp{desktop-task} pour afficher correctement le logo au fond et du texte,
  \item \codeLisp{force-draw} qui donne un rendu correct (en provoquant le dessin des fenêtres inactives qui ont pu être recouvertes) mais qui est très lent.
 \end{itemize}

\end{frame}

\begin{frame}
 \frametitle{Implémentation}
 
 \sourceLisp{shell.l}{227}{234}{Ajout de la tâche de bureau}
\end{frame}



\subsection{Analyse de la lenteur du rendu}

\begin{frame}
 \frametitle{Raisons}

 Il se trouve que le rendu des fenêtres d'\foreign{Interim OS} est très lent. En effet une simple remise à blanc de l'écran par \foreign{frame} impacte drastiquement les performances en ralentissant l'exécution de l'ordre de 5 fois. Ceci n'est pas étonnant car toutes les opérations se font pixel par pixel. 
 
 Il y a deux raisons à cela :
 \begin{itemize}
  \item les opérations graphiques devraient être gérées par le \foreign{GPU},
  \item pour chaque pixel, il y a un certain coût fixe lié à l'interfaçage entre \foreign{Minilisp} et le \codeC{Filesystem} représentant l'écran puis le programme \foreign{C}.
 \end{itemize}

\end{frame}

\begin{frame}
 \frametitle{Solution proposée}
 
 Ainsi pour résoudre ce problème, il faudrait ajouter des primitives pour remplir directement une zone entière d'une certaine couleur. Si on assimile l'écran à un carré de pixels de coté $n$, alors on divise ainsi le coût fixe par un facteur proportionnel à $n^2$ puisqu'on ne le paye désormais qu'une seule fois par rectangle au lieu d'une fois par pixel.
\end{frame}


\appendix

\tiny

\section{Annexe}

\subsection{Liste des fichiers}

\begin{frame}
 \frametitle{Fichiers \foreign{C}}
 
 \begin{table}[H]
  \centering
  \begin{tabularx}{\linewidth}{|C|C|}
    \hline
    Nom du fichier & Contenu \tabularnewline
    \hhline{|=|=|}
    \code{strmap.[h|c]} & Opérations sur une table de hachage dont les clés sont des chaînes de caractères \tabularnewline
    \hhline{|=|=|}
    \code{minilisp.h} & Contient la représentation mémoire des valeurs du \foreign{Minilisp} \tabularnewline
    \hhline{|=|=|}
    \code{alloc.[h|c]} & Définition de l'environnement, des différents tas, du \foreign{garbage collector} et des fonctions d'allocations des différents types de cellules \tabularnewline
    \hhline{|=|=|}
    \code{utf8.[h|c]} & Conversion entre chaînes de caractères standard et l'encodage \foreign{UTF-8} \tabularnewline
    \hline
    \code{reader.[h|c]} & \foreign{Parser} de \foreign{minilisp} \tabularnewline
    \hline
    \code{writer.[h|c]} & Fonctions pour écrire une valeur \foreign{Minilisp} dans un \foreign{buffer} \tabularnewline
    \hline
    \code{stream.[h|c]} & Représentations des systèmes de fichiers, et fonctions pour les ouvrir, fermer, écrire\ldots \tabularnewline
    \hhline{|=|=|}
    \code{jit\_x64.c} & Fonctions pour écrire de l'assembleur \foreign{x86-64} \tabularnewline
    \hline
    \code{compiler\_new.[h|c]} & Compile une expression \foreign{Minilisp} en assembleur, et initialise l'environnement avec les primitives \foreign{Minilisp} \tabularnewline
    \hline
    \code{compiler\_x64\_hosted.c} & Compile l'expression en assembleur \foreign{x86-64}, l'exécute et renvoie le résultat de son exécution \tabularnewline
    \hhline{|=|=|}
    \code{sledge.c} & Contient la fonction principale, qui ouvre un \foreign{channel} passé en argument, y lit une expression qu'il \foreign{parse} et exécute avant d'afficher le résultat \tabularnewline
    \hline
  \end{tabularx}
  \caption{Liste des fichiers \foreign{C}}
 \end{table}
\end{frame}

\begin{frame}
 \frametitle{Fichiers \foreign{Minilisp}}
 
 \begin{table}[H]
  \centering
  \begin{tabularx}{\linewidth}{|C|C|}
    \hline
    Nom du fichier & Contenu \tabularnewline
    \hhline{|=|=|}
    \code{lib.l} & Fonctions de base sur les listes et les chaînes de caractères \tabularnewline
    \hline
    \code{gfx.l} & Fonctions de base d'affichage de figures géométriques \tabularnewline
    \hhline{|=|=|}
    \code{mouse.l} & Gestion de la souris comme système de fichiers \tabularnewline
    \hline
    \code{net.l} & Communication sur internet (notamment par \foreign{IRC}) par un système de fichiers \tabularnewline
    \hhline{|=|=|}
    \code{editor.l} & Fonctionnement de l'éditeur : affichage, gestion des touches pressées, \ldots \tabularnewline
    \hline
    \code{repl.l} & Fonctionnement du \foreign{REPL (read-eval-print-loop)} : affichage, gestion de l'historique des commandes, \ldots \tabularnewline
    \hline
    \code{paint.l} & Application de dessin ? \tabularnewline
    \hhline{|=|=|}
    \code{shell.l} & Gestion des différentes tâches, ajout du logo, d'un éditeur et d'un \foreign{REPL} \tabularnewline
    \hline
  \end{tabularx}
  \caption{Liste des fichiers \foreign{Minilisp}}
 \end{table}
\end{frame}


\subsection{Liste des primitives Minilisp}

\begin{frame}
 \frametitle{Arithmétique et logique}
 
 \begin{table}[H]
  \centering
  \begin{tabularx}{\linewidth}{|C|C|}
    \hline
    Signature & Effet \tabularnewline
    \hhline{|=|=|}
    \codeLisp{(bitand a b)} & Et bit-à-bit \tabularnewline
    \hline
    \codeLisp{(bitnot a b)} & Non bit-à-bit \tabularnewline
    \hline
    \codeLisp{(bitor a b)} & Ou inclusif bit-à-bit \tabularnewline
    \hline
    \codeLisp{(bitxor a b)} & Ou exclusif bit-à-bit \tabularnewline
    \hhline{|=|=|}
    \codeLisp{(shl a b)} & Décalage logique vers la gauche \tabularnewline
    \hline
    \codeLisp{(shr a b)} & Décalage logique vers la droite \tabularnewline
    \hhline{|=|=|}
    \codeLisp{(+ a b)} & Addition \tabularnewline
    \hline
    \codeLisp{(- a b)} & Soustraction \tabularnewline
    \hline
    \codeLisp{(* a b)} & Multiplication \tabularnewline
    \hline
    \codeLisp{(/ a b)} & Quotient de la division \tabularnewline
    \hline
    \codeLisp{(mod a b)} & Reste de la division \tabularnewline
    \hhline{|=|=|}
    \codeLisp{(gt a b)} & Test de supériorité \tabularnewline
    \hline
    \codeLisp{(lt a b)} & Test d'infériorité \tabularnewline
    \hline
    \codeLisp{(= a b)} & Test d'égalité \tabularnewline
    \hline
  \end{tabularx}
  \caption{Liste des \foreign{builtins} \foreign{Minilisp} arithmético-logiques}
 \end{table}
\end{frame}

\begin{frame}
 \frametitle{Contrôle}
 
 \begin{table}[H]
  \centering
  \begin{tabularx}{\linewidth}{|C|C|}
    \hline
    Signature & Effet \tabularnewline
    \hhline{|=|=|}
    \codeLisp{(def x v)} & Définit globalement \codeLisp{x} comme valant \codeLisp{v} \tabularnewline
    \hline
    \codeLisp{(let x v)} & Définit \codeLisp{x} comme valant \codeLisp{v} (allocation locale sur la pile, qui pourra donc être \foreign{garbage-collectée}) \tabularnewline
    \hline
    \codeLisp{(fn x1 .. xn r)} & Renvoie une fonction à \code{n} arguments qui renvoie \codeLisp{r}, un argument est soit un symbole soit de la forme \codeLisp{(symb struct_def)}, et \codeLisp{symb} sera alors de type \codeLisp{struct_def} \tabularnewline
    \hhline{|=|=|}
    \codeLisp{(if b x y)} & Si \codeLisp{b} évalue à vrai, renvoie \codeLisp{x}, sinon \codeLisp{y} (qui doivent avoir le même type) \tabularnewline
    \hline
    \codeLisp{(while b e)} & Exécute \codeLisp{e} tant que \codeLisp{b} est vrai \tabularnewline
    \hline
    \codeLisp{(do x1 .. xn)} & Exécute \codeLisp{x1}, .., \codeLisp{xn} et renvoie \codeLisp{xn} \tabularnewline
    \hhline{|=|=|}
    \codeLisp{(car l)} & Renvoie la tête de la liste \codeLisp{l} \tabularnewline
    \hline
    \codeLisp{(cdr l)} & Renvoie la queue de la liste \codeLisp{l} \tabularnewline
    \hline
    \codeLisp{(cons x l)} & Renvoie la liste \codeLisp{(x l)} \tabularnewline
    \hline
    \codeLisp{(list x1 .. xn)} & Renvoie \codeLisp{(cons x1 (.. (cons xn-1 xn)..))} \tabularnewline
    \hhline{|=|=|}
    \codeLisp{(struct s c1 x1 .. cn xn)} & Définit \codeLisp{s} comme une structure contenant \code{n} champs dont les noms sont \codeLisp{fi} et les valeurs par défaut \codeLisp{xi} \tabularnewline
    \hline
    \codeLisp{(new s)} & Alloue et renvoie une structure de type \codeLisp{s} \tabularnewline
    \hline
    \codeLisp{(sget s c)} & Renvoie la valeur du champ \codeLisp{c} de la structure \codeLisp{s} \tabularnewline
    \hline
    \codeLisp{(sput s c v)} & Affecte la valeur \codeLisp{v} au champ \codeLisp{c} de la structure \codeLisp{s} \tabularnewline
    \hline
  \end{tabularx}
  \caption{Liste des \foreign{builtins} \foreign{Minilisp} de contrôle}
 \end{table}
\end{frame}

\begin{frame}
 \frametitle{Mémoire}
 
 \begin{table}[H]
  \centering
  \begin{tabularx}{\linewidth}{|C|C|}
    \hline
    Signature & Effet \tabularnewline
    \hhline{|=|=|}
    \codeLisp{(quote x)} & Renvoie l'adresse du symbole \codeLisp{x} \tabularnewline
    \hline
    \codeLisp{(concat s t)} & Renvoie la concaténation des chaînes \codeLisp{s} et \codeLisp{t} \tabularnewline
    \hline
    \codeLisp{(substr s a b)} & Renvoie une copie de la chaîne \codeLisp{s} entre \codeLisp{a} et \codeLisp{b} \tabularnewline
    \hhline{|=|=|}
    \codeLisp{(get8 s i)} & Renvoie l'octet de la chaîne \codeLisp{s} situé en position \codeLisp{i} \tabularnewline
    \hline
    \codeLisp{(get16 s i)} & Renvoie deux octets de la chaîne \codeLisp{s} à partir de \codeLisp{i} \tabularnewline
    \hline
    \codeLisp{(get32 s i)} & Renvoie quatre octets de la chaîne \codeLisp{s} à partir \codeLisp{i} \tabularnewline
    \hline
    \codeLisp{(put8 s i v)} & Modifie l'octet de la chaîne \codeLisp{s} situé en position \codeLisp{i} \tabularnewline
    \hline
    \codeLisp{(put16 s i v)} & Modifie deux octets de la chaîne \codeLisp{s} à partir de \codeLisp{i} \tabularnewline
    \hline
    \codeLisp{(put32 s i v)} & Modifie quatre octets de la chaîne \codeLisp{s} à partir \codeLisp{i} \tabularnewline
    \hhline{|=|=|}
    \codeLisp{(alloc n)} & Alloue et renvoie \codeLisp{n} octets \tabularnewline
    \hline
    \codeLisp{(alloc_str n)} & Alloue et renvoie une chaîne de \codeLisp{n} caractères \tabularnewline
    \hline
    \codeLisp{(bytes_to_str b n)} & Alloue et renvoie une chaîne de \codeLisp{n} caractères obtenues depuis \codeLisp{b} \tabularnewline
    \hhline{|=|=|}
    \codeLisp{(size x)} & Renvoie la taille de \codeLisp{x} en mémoire \tabularnewline
    \hline
    \codeLisp{(gc)} & Appelle la fonction \codeC{collect_garbage} \tabularnewline
    \hline
    \codeLisp{(symbols)} & Renvoie la liste des symboles connus \tabularnewline
    \hline
    \codeLisp{(debug)} & Devrait appeler \codeC{platform_debug} mais a été commenté \tabularnewline
    \hline
  \end{tabularx}
  \caption{Liste des \foreign{builtins} \foreign{Minilisp} de gestion de la mémoire}
 \end{table}
\end{frame}

\begin{frame}
 \frametitle{Fichiers}
 
 \begin{table}[H]
  \centering
  \begin{tabularx}{\linewidth}{|C|C|}
    \hline
    Signature & Effet \tabularnewline
    \hhline{|=|=|}
    \codeLisp{(write x b)} & Écrit la représentation de \codeLisp{x} dans le \foreign{buffer} \codeLisp{b} \tabularnewline
    \hline
    \codeLisp{(read b)} & Lit le code \foreign{Minilisp} situé dans le \foreign{buffer} \codeLisp{b} et renvoie une valeur \foreign{Minilisp} correspondante \tabularnewline
    \codeLisp{(eval x)} & Exécute le code situé dans \codeLisp{x} (typiquement renvoyé par \codeLisp{read}) en appelant \codeC{platform_eval} \tabularnewline
    \hhline{|=|=|}
    \codeLisp{(print s)} & Affiche la (liste de) chaîne(s) de caractères \codeLisp{s} \tabularnewline
    \hline
    \codeLisp{(mount p h)} & Monte le fichier à l'emplacement \codeLisp{p}, où \codeLisp{h} est une liste de fonctions permettant d'opérer sur le fichier (inutilisé) \tabularnewline
    \hline
    \codeLisp{(mmap p)} & Applique l'opérateur \codeLisp{mmap} du système de fichier considéré au fichier à l'emplacement \codeLisp{p} \tabularnewline
    \hline
    \codeLisp{(open p)} & Ouvre le fichier à l'emplacement \codeLisp{p} \tabularnewline
    \hline
    \codeLisp{(recv f)} & Lit le fichier \codeLisp{f} \tabularnewline
    \hline
    \codeLisp{(send f s)} & Écrit la (liste de) chaîne(s) de caractères \codeLisp{s} dans le fichier \codeLisp{f} \tabularnewline
    \hline
  \end{tabularx}
  \caption{Liste des \foreign{builtins} \foreign{Minilisp} de gestion des fichiers}
 \end{table}
\end{frame}

\end{document}
